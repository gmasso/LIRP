\section{Strengthening the formulation}
\subsection{A difference approach}
\begin{table}[htbp]
\centering
\begin{tabular}{ll}
\toprule
$f_j$ & Fixed cost of opening distribution center $j\in\loc \setminus \cu$\\ 
$d_{it}$ & Demand of customer $i\in\cu$ in period $t \in T$\\
$h^{it}_{rjs}$ & Unitary holding cost for units serving $(i,t)$ arriving at location $j$ in period $s\leq t$ through route $r$\\
        $c_r$ & Cost of route $r\in \ro$\\
        $\alpha(r)$ & Set of facilities $i$ delivered by route $r$\\
        $\beta(r)$ & Index of the starting DC of route $r$\\
        $Q_k$ & Capacity of vehicles delivering locations at level $k$\\ 
        $\nu_k$ & Fleet size for vehicles delivering locations of level $k$\\ 
        \bottomrule
        \end{tabular}
        \caption{Data}
        \label{tab:data2}
        \end{table}       

        To ease the exposition of the modified formulation, we introduce some additional sets, which consists of pairs of indices from the sets defined in~\ref{tab:set}.
        We denote $\mathcal{D}$ the set of all demands that must be satisfied over the entire planning horizon:
        \begin{equation*}
            \mathcal{D} =\left\{(i,t) : i\in\cu, t=1,\ldots,T\right\}
        \end{equation*}
        Given a period $s=1,\ldots,T$, we also define $\mathcal{D}(s)$ as the set of all demands occuring after period $s$:
        \begin{equation*}
            \mathcal{D}(s) =\left\{(i,t) : i\in\cu, t=s,\ldots,T\right\}
        \end{equation*}
        In particular, one can note that $\mathcal{D}(1) = \mathcal{D}$
    Finally, we define for all $k=1,\ldots,L-1$ the set $\mathcal{S}_k$ of all pairs of routes in level $k$ with one of their stops. That is, for $k=1,\ldots,L$,
        \begin{equation*}
            \mathcal{S}_k =\left\{(r,j) : r\in\ro_k, j\in\alpha(r)\right\}
        \end{equation*}
        \begin{table}[htbp]
        \centering
        \begin{tabular}{ll}
        \toprule
        \multicolumn{2}{l}{\textit{Binary Variables}}\\
            $y_{j}$ & $=1$ if distribution center $j \in \loc \backslash \cu$ is selected \\
            $z_{rt}$ & $=1$ if route $r \in \ro$  is selected in period $t \in T$\\
            \midrule
            \multicolumn{2}{l}{\textit{Continuous variables}}\\
                $x^{it}_{rjs}\in[0,1]$ & Proportion of demand $(i,t)$ delivered to location $j\in \loc$ in period $s\leq t$\\
                \bottomrule
                \end{tabular}
                \caption{Variables}
                \label{tab:var2}
                \end{table}    

                %Partition the demand points in time and space. Accessible demand points are on a given route starting from a given source, and only the future periods are considered.
                \begin{alignat}{3}
                    \text{minimize} &&\sum_{j\in \loc \setminus \cu} f_{j} y_{j} +\sum_{s=1}^{T}  \sum_{r\in \ro} \left(c_r z_{rs} + \sum_{(i,t)\in\mathcal{D}(s)} \sum_{j\in\alpha(r)} h^{it}_{rjs}x^{it}_{rjs}\right)\span\span\label{objfunct2}\\ 
    \text{s.t.}  %&&\sum_{r\in \ro} \gamma_{jr} z_{rt} &\leq 1 															&\forall j&\in \, \forall t=1,\ldots,T  \label{const:singleroutedepots}\\
    &&\sum_{r : i\in\alpha(r)} z_{rs} &\leq 1 															&\forall i&\in \cu, \forall s=1,\ldots,T  \label{const:singleroutecustomers2}\\
    &&\sum_{r : j\in\alpha(r)} z_{rs} &\leq y_j 															&\forall j&\in \loc\setminus \cu, \forall s=1,\ldots,T  \label{const:singleroutedc2}\\
    %&&\gamma_{jr}z_{rt} 					&\leq y_j 													&\forall k&=1,\ldots,L, \forall j\in\loc_{k-1}, \forall r\in \ro_k, \forall t=1,\ldots,T\label{const:serveopendepots}\\
    &&z_{rs} 					&\leq y_{\beta(r)} 													&\forall r&\in \ro, \forall s=1,\ldots,T\label{const:startfromopendepots2}\\
    &&\sum_{r\in \ro_k} z_{rs} &\leq 	\nu_k													&\forall k&=1,\ldots,L, \forall s=1,\ldots,T  \label{const:fleetcapa2}\\
    %            &&\sum_{r\in \roc} z_{rt} &\leq 	\nu_{\cu}													&\forall t&=1,\ldots,T  \label{const:fleetcapaclients}\\
    &&\sum_{s=1}^t \sum_{(r,i)\in \mathcal{S}_L}x^{it}_{ris}   		& = 1 														&\forall (i,t)&\in\mathcal{D}\label{const:satdemands}\\
    &&x^{it}_{rjs}   		&\leq  z_{rs}													&\forall (i,t)&\in\mathcal{D}, s=1,\ldots,t,k=1,\ldots,L, (r,j)\in\mathcal{S}_k\label{const:deliveryConstLoc2}\\
    &&\sum_{(i,t)\in\mathcal{D}(s)}\sum_{j\in\alpha(r)} d_{it} x^{it}_{rjs}   		&\leq Q_k z_{rs} 														&\forall k&=1,\ldots,L, \forall r\in \ro_k, s=1,\ldots,T\label{const:deliveryUB2}\\
    %&&\sum_{i\in \cu} u^t_{ir}   		&\leq \kc z_{rt} 														&\forall r&\in \roc, t=1,\ldots,T\label{const:deliveryUBcustomers}\\
    &&\sum_{u=1}^s \left(\sum_{(r,j)\in\mathcal{S}_k} x^{it}_{rjs} - \sum_{(r',j')\in\mathcal{S}_{k+1}} x^{it}_{r'j's}\right) &\geq 0\quad 			&\forall k&=1,\ldots,L-1, \forall i\in\cu, \forall t=1,\ldots,T \label{const:enoughunits}\\
 %   &&I_{jt}					&\leq I_j^{\max} y_j  									&\forall k&=1,\ldots,L-1, \forall j\in \loc_k,\forall t=1,\ldots,T\label{const:invdepotUB2}\\	
 %   &&I_{it} 					&\leq \min\left(I_i^{\max}, \sum_{t' > t}d_{it'}\right)											&\forall i&\in \cu, \forall t=1,\ldots,T\label{const:invcustUB2}\\
    &&x^{it}_{rjs}			&\geq 0 															&\forall i&\in \cu,\forall t=1,\ldots, T, \forall j\in \loc\setminus\cu\cup\{i\}, \forall s= 1,\ldots t\label{const:upos2}	\\
    &&y_{j}					& \in \{0,1\} 														&\forall j&\in \loc\setminus\cu\label{const:ybool2}\\	
    &&z_{rs}					&\in \{0,1\} 														&\forall r&\in \ro, \forall s=1,\ldots,T\label{const:zbool2}
\end{alignat}

Constraint~\eqref{const:singleroutecustomers2} ensures that a customer is served by at most one route in each period.
Constraint~\eqref{const:singleroutedc2} ensures that a dc is served by at most one route in each period.
Constraint~\eqref{const:startfromopendepots2} ensures that each active route in period $s$ start from an open dc.
Constraint~\eqref{const:fleetcapa2} ensures that there are no more active routes than the number of vehicles in period $s$.
Constraint~\eqref{const:satdemands} ensures that all demands are satisfied.
Constraint~\eqref{const:deliveryConstLoc2} ensures that units can be delivered to location $j$ in period $s$ only if it belongs to an active route in this period.
Constraint~\eqref{const:deliveryUB2} ensures that a vehicle using a route $r$ in period $s$ do not deliver more units than its capacity.
Constraint~\eqref{const:enoughunits} ensures that units delivered to the locations in a given layer are present on the upper level of the network.

\subsection{Valid flow cover inequalities}
In our numerical experiments we shall compare the results obtained using the first formulation versus the second one to determine whether this one is stronger and gives better lower bounds.

In addition, this latter allows us to add the so-called flow-cover inequalities as cuts to strengthen the LP formulation and obtain an even tighter lower bound.
We use an approach similar than the one presented in~\cite{levi_approximation_2008}.
Let $A$ be a subset of demand points $(i,t)\in\mathcal{D}$ and $D(A)=\sum_{(i,t)\in A} d_{it}$ the cumulative demand of $A$.
Let $l^k_A=\lceil D(A)/Q_k \rceil$ be the {\em cover number of $A$ at level $k$}, $k=1,\ldots,L$, i.e. the minimum number of vehicles of level $k$ required to satisfy the demands in A. 
Let $\lambda^k_A = l^k_A Q_k - D(A)$ be the capacity left in $l^k_A$ vehicles if they carry $D(A)$ units and let $R^k_A = Q_k - \lambda^k_A$ be the {\em residual capacity of $A$ at level $k$}, i.e. the capacity required to satisfy the demands in $A$ after $l^k_A - 1$ vehicles of level $k$ are fully used.
Finally let $\rho^k_A = R^k_A/Q_k$ ($= 1 - l^k_A + D(A)/Q_k$) be the {\em fraction of the residual capacity at level $k$}. 
Observe that by definition $0<R^k_A \leq Q_k$ and $0<\rho^k_A\leq 1$. 

For each level $k=1,\ldots,L$ and $A$ a subset of demand points, we call a subset $F$ of periods (i.e., $F\subseteq \left\{1,\ldots,T\right\}$) a {\em cover of $A$ at level $k$} if $|F|\geq l^k_A$.
We now present the route-based version of the flow-cover inequalities. 
Let $A$ be a subset of demand points, $k=1,\ldots, L$ and let $F$ be a cover of $A$ at level $k$. First, note that 
\begin{align}
    D(A) &= \sum_{(i,t)\in A} \sum_{(r,j)\in\mathcal{S}_k} \sum_{s\in F} d_{it}x^{it}_{rjs}+  \sum_{(i,t)\in A}\sum_{s\notin F}\sum_{(r,j)\in\mathcal{S}_k} d_{it}x^{it}_{rjs}\nonumber\\
         &\leq Q_k \sum_{r\in\ro_k}\sum_{s\in F} z_{rs}+  \sum_{(i,t)\in A}\sum_{(r,j)\in\mathcal{S}_k}\sum_{s\notin F} d_{it}x^{it}_{rjs}\label{ineq:ineqdemA_inter}\\
    \frac{D(A)}{Q_k}&\leq \sum_{r\in\ro_k}\sum_{s\in F} z_{rs}+  \frac{1}{Q_k} \sum_{(i,t)\in A}\sum_{(r,j)\in\mathcal{S}_k}\sum_{s\notin F} d_{it}x^{it}_{rjs}\label{ineq:ineqdemA}
\end{align}
, where the inequality~\eqref{ineq:ineqdemA_inter} comes from constraint~\eqref{const:deliveryUB2}. 
Since $z_{rs}$ are integers and $x^{it}_{rjs}$ are continuous, one can derive {\em Mixed Integer Rounding} (MIR) inequalities to obtain valid cuts for the LP formulation based on flows.
MIR inequalities apply to mixed-integer sets of the form $\mathcal{Q} = \{x\in\mathbb{R}, y\in\mathbb{Z} : x+y\geq b, x\geq 0\}$. 
Specifically, one can prove that the inequality $x+(b-\lfloor b\rfloor) y \geq (b-\lfloor b \rfloor)\lceil b\rceil$ is valid. 
It is interesting to note that values of $y$ in the convex hull of $\mathcal{Q}$ that violate the MIR inequality fall within $(\lfloor b\rfloor, \lceil b \rceil)$.
This can be generalized to more complicated sets that involve more variables, as long as the variables can be split into an integral part and a continuous nonnegative part. 
In particular, we apply an MIR derivation to the mixed-integer set defined by inequalities~\eqref{ineq:ineqdemA} and obtain the following inequalities for all $k=1,\ldots,L$:
\begin{equation}
    \sum_{r\in\ro_k} \left(\rho^k_A  \sum_{s\in F} z_{rs}+  \frac{1}{Q_k} \sum_{(i,t)\in A}\sum_{j\in\alpha(r)}\sum_{s\notin F} d_{it}x^{it}_{rjs}\right)\geq \rho^k_A \left\lceil\frac{D(A)}{Q_k}\right\rceil   \label{ineq:FCCut}
\end{equation}

Finally, note that at level $L$, the set of locations eligible to collect units to serve demands of client $i$ is precisely reduced to the singleton $\{i\}$ and therefore inequality~\eqref{ineq:FCCut} simplifies into:
\begin{equation}
   \sum_{r\in\ro_L} \rho^L_A  \sum_{s\in F} z_{rs}+  \frac{1}{Q_L} \sum_{(i,t)\in A}\sum_{r:i\in\alpha(r)}\sum_{s\notin F} d_{it}x^{it}_{ris}\geq \rho^L_A \left\lceil\frac{D(A)}{Q_L}\right\rceil   \label{ineq:FCCutClient}
\end{equation}
