\documentclass[10pt, a4paper]{article}
\usepackage[margin=2cm]{geometry}

%%% Language and encoding %%%
\usepackage[utf8]{inputenc}
\usepackage[T1]{fontenc}
%%% Mathematical symbols and notations %%%
\usepackage{amsthm, amsmath, mathrsfs, amssymb}
\usepackage{textcomp}
\usepackage{algorithm}
\usepackage{algorithmic}

%%% Figures %%%
\usepackage{caption, subcaption}
\usepackage{tikz}
\usepackage{tkz-graph}
\usetikzlibrary{shapes, shapes.gates.logic.US, trees, positioning, arrows, decorations.pathreplacing, plotmarks, backgrounds,shapes}
%%% Other packages %%%
\usepackage[affil-it]{authblk}
\usepackage{multirow}
\usepackage{xcolor}
\definecolor{bleuclairimtatlantique}{RGB}{0,184,222}
\definecolor{noirimtatlantique}{RGB}{60, 60, 60}
\definecolor{blancimtatlantique}{RGB}{255, 255, 255}
\definecolor{bleufonceimtatlantique}{RGB}{12, 35, 64}
\definecolor{vertimtatlantique}{RGB}{164, 210, 51}
\definecolor{grisimtatlantique}{RGB}{230, 230, 230}
\definecolor{ocre}{RGB}{250,82,20}
\usepackage{url}
\usepackage{hyperref}
\usepackage{verbatim}
\usepackage{varwidth}
%% Define the figures path
\newcommand{\inputfig}[1]{\input{figures/#1}}
\graphicspath{{figures/}}

%\usepackage{empheq}
\usepackage[framemethod=tikz]{mdframed}

%%%%% Definition of commands %%%%%
\newcommand{\loc}{\mathcal{L}}
\newcommand{\cu}{\mathcal{C}}
\newcommand{\su}{\mathcal{S}}
\newcommand{\ro}{\mathcal{R}}

%%%%% Definitions for tikz %%%%%
% Definition of styles for diagrams :
\tikzset{%
    % Symbols for block diagrams
    block/.style    = {draw, thick, rectangle, minimum height = 3em,
    minimum width = 3em},
    sum/.style      = {draw, circle, node distance = 2cm}, % Adder
    input/.style    = {coordinate}, % Input
    output/.style   = {coordinate}, % Output
    % Gates and symbols style for the fault trees
    and/.style={and gate US,thick,draw,fill=red!20,rotate=90,
    anchor=east,xshift=-1mm},
    or/.style={or gate US,thick,draw,fill=blue!20,rotate=90,
    anchor=east,xshift=-1mm},
    be/.style={circle,thick,draw,fill=green!20,anchor=north,
    minimum width=0.7cm},
    tr/.style={buffer gate US,thick,draw,fill=purple!40,rotate=90,
    anchor=east,minimum width=0.8cm},
    % Label style
    label distance=3mm,
    every label/.style={blue},
    % Event style
    event/.style={rectangle,thick,draw,fill=yellow!10,text width=2cm,
    text centered,font=\sffamily,anchor=north},
    % Children and edges style
    edge from parent/.style={very thick,draw=black!70},
    edge from parent path={(\tikzparentnode.south) -- ++(0,-1.05cm)
    -| (\tikzchildnode.north)},
    level 1/.style={sibling distance=7cm,level distance=1.4cm,
    growth parent anchor=south,nodes=event},
    level 2/.style={sibling distance=7cm},
    level 3/.style={sibling distance=6cm},
    level 4/.style={sibling distance=3cm},
    %%  For compatability with PGF CVS add the absolute option:
    absolute
}

\tikzstyle{client}=[regular polygon,regular polygon sides=3,draw,fill=blue!50,fill opacity=1,scale=0.4]
\tikzstyle{ptf}=[draw,circle,fill=red!50,fill opacity=1,scale=0.8]
\tikzstyle{cdc}=[regular polygon, regular polygon sides=3,draw,fill=yellow!70,fill opacity=1,scale=1.2]

\tikzstyle{legend}=[scale=0.7,right,text justified]
\tikzstyle{shipment}=[->,>=stealth',blue]
\tikzstyle{ftlarc}=[->,>=stealth',very thick,red]
\tikzstyle{vertexD}	=[circle,draw=black!80,fill=purple!70,minimum size=7pt,inner sep=0pt]
\tikzstyle{vertex0}	=[circle,draw=black!80,fill=purple!20,minimum size=7pt,inner sep=0pt]
\tikzstyle{j0}			=[circle, draw=black,fill=purple,minimum size=13pt,inner sep=0pt, scale=1]
\tikzstyle{farm}		=[circle, draw=black,fill=blue!80,minimum size=4pt,inner sep=0pt, scale=1]

\tikzstyle{com2}		= [minimum size=0pt,inner sep=0pt, scale=1]
\tikzstyle{edgetr} 	= [draw,thick,->,purple]
\tikzstyle{edge1} 	= [draw,dotted,-,blue, thick]



\title{Location Inventory Routing Problem}
\author[1]{Guillaume Massonnet\thanks{guillaume.massonnet@imt-atlantique.fr}}
\author[1]{Olivier Péton\thanks{olivier.peton@imt-atlantique.fr}}
\affil[1]{IMT Atlantique, LS2N, 4 rue Alfred Kastler, 44300 Nantes, France}

\begin{document}
\maketitle
\begin{abstract}                % Abstract of not more than 250 words.
The abstract here. 
\end{abstract}

%\begin{keyword}
%Five to ten keywords, preferably chosen from the IFAC keyword list.
%\end{keyword}

\section{Introduction}
Location Inventory Routing Problem (LIRP) aims at integrating in a unified model the three level of decisions involved in the optimization of the supply chain. 
The strategical aspect relates to the designe of the network, by positioning and opening distribution centers on a geographical area.
At the tactical, production and inventory management decisions are made while the operational part of the problem focuses on logistics and transportation solutions. 

\subsection{Literature review}
Fill the .bib file and reference the papers, explain what the authors are doing.
\cite{Aziz17}

\section{The model}
\subsection{Description and notations}
In this work, we present a unified model that covers a broad spectrum of systems in which one supplier (or central plant) produces semi-finished or end products. 
There is a set of candidates depots, that is location onto which {\em distribution centers} (DC), or {\em depots} may be opened to operate as intermediate between the central supplier and a (often larger) set of retailers, also called customers.
The decision to open or not a DC incurs a fixed opening cost for each depot at the beginning of the planning horizon and cannot be changed during the exploitation of the network.
Units stored at a DC or at a customer incur a holding cost that represents the financial immobilisation of the good as well as maintenance costs.
Finally deliveries are made through predefined routes through the network. A route starts either from the supplier and visits a set of depots before returning to the supplier or starts from a depot and visits a set of customers before returning to the same depot. 
Using a route to deliver units during a time period also incurs a transportation cost that consists of a (possibly zero) fixed ordering cost along with a routing cost that depends on the length of the route and its number of stops.

Throughout the paper, we use the following notations:
\begin{table}
    \centering
    \begin{tabular}{ll}
        \hline
        Set & Definition \\
        \hline
        $T$ & Length of the planning horizon\\
        $L$ & Number of levels in the distribution network\\
        $\loc_k$ & Set of distribution centers $j$ of level $k=1, \ldots,L$ \\
        $\loc=\cup_{k=1}^L \loc_k$ & Set of all locations in the distribution network\\
        $\cu=\loc_L$ & Set of customers \\ 
        $\ro_k$ & Set of routes starting from a DC at level $k-1$ and visiting and a subset of locations at level $k$, $k=1,\ldots,L$\\
        $\ro = \cup_{k=1}^{L} \ro_k$ & Set of all the routes in the distribution network\\
        \hline
        \hline
        Data & Definition \\
        \hline
        $f_j$ & Fixed cost of opening distribution center $j\in\loc_k$, $k=1,\ldots,L-1$\\ 
        $\kappa_k$ & Capacity of vehicles delivering locations at level $k$\\ 
        $\nu_k$ & fleet size for vehicles delivering locations of level $k$\\ 
        $d_{it}$ & Demand faced by customer $i\in\cu$ in period $t=1,\ldots,T$\\
        $h_{it}$ & Per-unit, per-period holding cost of location $i\in\loc$ in period $t=1,\ldots, T$\\
        $I_{i0}$ & Initial inventory of location $i\in \loc$ at the beginning of the planning horizon\\
        $c_r$ & Cost of route $r\in \ro$\\
        $\alpha_{ir}$ & indicator that route $r$ visits location $i$\\
        $\beta_{jr}$ & indicator that route $r$ starts from DC $j$\\
        $I_i^{\max}$ & Maximum inventory at location $i\in\loc$\\
        \hline
        \hline
        \multicolumn{2}{l}{Variables}\\
        \hline
        \multicolumn{2}{l}{\textit{Binary Variables}}\\
        $y_{j}$ & $\rightarrow$ 1 if distribution center $j$ is selected \\
        $z_{rt}$ & $\rightarrow$ 1 if route $r$ from level is selected in period $t$\\
        \hline
        \multicolumn{2}{l}{\textit{Continuous variables}}\\
        $q_{irt}$ & $\rightarrow$ quantity delivered by route $r$ to location $i$ in period $t=1,\ldots,T$\\
        $I_{it}$ & $\rightarrow$ Inventory at location $i$ in period $t=1,\ldots,T$\\
        \hline
    \end{tabular}
\end{table}

\subsection{MIP formulation}\label{subsection:MIP}
\begin{alignat}{3}
    \text{minimize} &&\sum_{j\in \loc \setminus \cu} f_{j} y_{j} &+\sum_{t=1}^{T} \left( \sum_{r\in \ro} c_r z_{rt} + \sum_{i\in \loc} h_{it} I_{it}\right)\label{objfunct}\\ 
    \text{s.t.}  %&&\sum_{r\in \ro} \gamma_{jr} z_{rt} &\leq 1 															&\forall j&\in \, \forall t=1,\ldots,T  \label{const:singleroutedepots}\\
    &&\sum_{r\in \ro_L} \alpha_{ir} z_{rt} &\leq 1 															&\forall i&\in \cu, \forall t=1,\ldots,T  \label{const:singleroutecustomers}\\
&&\sum_{r\in \ro_k} \alpha_{jr} z_{rt} &\leq y_j 															&\forall k&=1,\ldots,L-1, \forall j\in \loc_k, \forall t=1,\ldots,T  \label{const:singleroutedc}\\
    %&&\gamma_{jr}z_{rt} 					&\leq y_j 													&\forall k&=1,\ldots,L, \forall j\in\loc_{k-1}, \forall r\in \ro_k, \forall t=1,\ldots,T\label{const:serveopendepots}\\
&&z_{rt} 					&\leq \sum_{j\in\loc_{k}}\beta_{jr}y_j 													&\forall k&=1,\ldots,L-1,r\in \ro_{k+1}, \forall t=1,\ldots,T\label{const:startfromopendepots}\\
    &&\sum_{r\in \ro_k} z_{rt} &\leq 	\nu_k													&\forall k&=1,\ldots,L, \forall t=1,\ldots,T  \label{const:fleetcapa}\\
    %            &&\sum_{r\in \roc} z_{rt} &\leq 	\nu_{\cu}													&\forall t&=1,\ldots,T  \label{const:fleetcapaclients}\\
    &&\sum_{i\in \loc_k} q_{irt}   		&\leq \kappa_k z_{rt} 														&\forall k&=1,\ldots,L, \forall r\in \ro_k, t=1,\ldots,T\label{const:deliveryUB}\\
    &&q_{irt}   		&\leq \alpha_{ir}\kappa_k  														&\forall k&=1,\ldots,L, \forall r\in \ro_k, t=1,\ldots,T\label{const:deliveryConstLoc}\\
    %&&\sum_{i\in \cu} u^t_{ir}   		&\leq \kc z_{rt} 														&\forall r&\in \roc, t=1,\ldots,T\label{const:deliveryUBcustomers}\\
    &&I_{j,t-1} + \sum_{r\in\ro_k} q_{jrt}   			&=I_{jt} +\sum_{r'\in \ro_{k+1}}\left(\beta_{jr'} \sum_{i\in \loc_{k+1}}q_{ir't}\right)\quad 			&\forall k&=1,\ldots,L-1, \forall j\in \loc_k, \forall t=1,\ldots,T \label{const:invflowdepots}\\
    &&I_{i,t-1}+\sum_{r\in \ro_L} q_{irt} &= I_{it} +d_{it} 							&\forall i&\in \cu, \forall t=1,\ldots,T\label{const:invflowcustomers}\\
    &&I_{jt}					&\leq I_j^{\max} y_j  									&\forall k&=1,\ldots,L-1, \forall j\in \loc_k,\forall t=1,\ldots,T\label{const:invdepotUB}\\	
    &&I_{it} 					&\leq \min\left(I_i^{\max}, \sum_{t' > t}d_{it'}\right)											&\forall i&\in \cu, \forall t=1,\ldots,T\label{const:invcustUB}\\
    &&q_{irt}			&\geq 0 															&\forall k&=1,\ldots,L, \forall i\in \loc_k, \forall r\in \ro_k,\forall t=1,\ldots, T\label{const:upos}	\\
    % &&v^t_{jr}			&\geq 0 															&\forall j &\in \dc, \forall r\in \rod,\forall t=1,\ldots, T\label{const:vpos}	\\
    &&I_{it}			&\geq 0 															&\forall i &\in \loc, \forall t=1,\ldots, T\label{const:invCpos}	\\
    %&&I^t_{j}			&\geq 0 															&\forall j &\in \dc, \forall t=1,\ldots, T\label{const:invDpos}	\\
    &&y_{j}					& \in \{0,1\} 														&\forall j&\in \loc\setminus\cu\label{const:ybool}\\	
    &&z_{rt}					&\in \{0,1\} 														&\forall r&\in \ro, \forall t=1,\ldots,T\label{const:zbool}
\end{alignat}

The objective function~\eqref{objfunct} aims at minimizing the total cost incurred by the system. 
Constraints~\eqref{const:singleroutecustomers} and~\eqref{const:singleroutedc} state that every (open) location is served by at most one route $r\in\ro$ serving the corresponding level in every period, respectively. 
Constraint~\eqref{const:startfromopendepots} ensures that routes start only from opened depots. 
Constraints~\eqref{const:fleetcapa} states that the number of routes used to serve the locations of a given level $k$ in period $t$ does not exceed the fleet size allocated to the routes $r\in\ro_k$.
Constraint~\eqref{const:deliveryUB} ensures that the sum of the quantities deliverd through a route $r\in\ro_k$ in period $t$ is lower than the capacity of a vehicle. 
Constraints~\eqref{const:invflowdepots} and~\eqref{const:invflowcustomers} define the units flow through the depots and customers, respectively.
Finally, constraints~\eqref{const:invdepotUB} and~\eqref{const:invcustUB} ensure that the capacity constraint on the inventory at the depots and the customers are satisfied in any period. 

\subsection{Another formulation}
Partition the demand points in time and space. Accessible demand points are on a given route starting from a given source, and only the future periods are considered.
\begin{alignat}{3}
    \text{minimize} &&\sum_{j\in \loc \setminus \cu} f_{j} y_{j} &+\sum_{t=1}^{T} \left( \sum_{r\in \ro} c_r z_{rt} + \sum_{i\in \loc} h_{it} I_{it}\right)\label{objfunct2}\\ 
    \text{s.t.}  %&&\sum_{r\in \ro} \gamma_{jr} z_{rt} &\leq 1 															&\forall j&\in \, \forall t=1,\ldots,T  \label{const:singleroutedepots}\\
    &&\sum_{r\in \ro_L} \alpha_{ir} z_{rt} &\leq 1 															&\forall i&\in \cu, \forall t=1,\ldots,T  \label{const:singleroutecustomers2}\\
&&\sum_{r\in \ro_k} \alpha_{jr} z_{rt} &\leq y_j 															&\forall k&=1,\ldots,L-1, \forall j\in \loc_k, \forall t=1,\ldots,T  \label{const:singleroutedc2}\\
    %&&\gamma_{jr}z_{rt} 					&\leq y_j 													&\forall k&=1,\ldots,L, \forall j\in\loc_{k-1}, \forall r\in \ro_k, \forall t=1,\ldots,T\label{const:serveopendepots}\\
&&z_{rt} 					&\leq \sum_{j\in\loc_{k}}\beta_{jr}y_j 													&\forall k&=1,\ldots,L-1,r\in \ro_{k+1}, \forall t=1,\ldots,T\label{const:startfromopendepots2}\\
    &&\sum_{r\in \ro_k} z_{rt} &\leq 	\nu_k													&\forall k&=1,\ldots,L, \forall t=1,\ldots,T  \label{const:fleetcapa2}\\
    %            &&\sum_{r\in \roc} z_{rt} &\leq 	\nu_{\cu}													&\forall t&=1,\ldots,T  \label{const:fleetcapaclients}\\
    &&\sum_{i\in\cu}\sum_{t=s}^T\sum_{j\in \loc_k} d_{it} x^{it}_{jrs}   		&\leq \kappa_k z_{rs} 														&\forall k&=1,\ldots,L, \forall r\in \ro_k, s=1,\ldots,T\label{const:deliveryUB2}\\
    &&x^{it}_{jrs}   		&\leq \alpha_{jr}  														&\forall k&=1,\ldots,L, \forall r\in \ro_k, i\in \cu, t=1,\ldots,T, s=1,\ldots,t\label{const:deliveryConstLoc2}\\
    %&&\sum_{i\in \cu} u^t_{ir}   		&\leq \kc z_{rt} 														&\forall r&\in \roc, t=1,\ldots,T\label{const:deliveryUBcustomers}\\
    &&I_{j,t-1} + \sum_{r\in\ro_k} q_{jrt}   			&=I_{jt} +\sum_{r'\in \ro_{k+1}}\left(\beta_{jr'} \sum_{i\in \loc_{k+1}}q_{ir't}\right)\quad 			&\forall k&=1,\ldots,L-1, \forall j\in \loc_k, \forall t=1,\ldots,T \label{const:invflowdepots2}\\
    &&I_{i,t-1}+\sum_{r\in \ro_L} q_{irt} &= I_{it} +d_{it} 							&\forall i&\in \cu, \forall t=1,\ldots,T\label{const:invflowcustomers2}\\
    &&I_{jt}					&\leq I_j^{\max} y_j  									&\forall k&=1,\ldots,L-1, \forall j\in \loc_k,\forall t=1,\ldots,T\label{const:invdepotUB2}\\	
    &&I_{it} 					&\leq \min\left(I_i^{\max}, \sum_{t' > t}d_{it'}\right)											&\forall i&\in \cu, \forall t=1,\ldots,T\label{const:invcustUB2}\\
    &&q_{jrt} &= \sum_{i\in\cu}\sum_{t=1}^T d_{it} x^{it}_{jrt} 															&\forall k&=1,\ldots,L, \forall i\in \loc_k, \forall r\in \ro_k,\forall t=1,\ldots, T\label{const:upos2}	\\
    &&x^{it}_{jrt}			&\geq 0 															&\forall k&=1,\ldots,L, \forall i\in \loc_k, \forall r\in \ro_k,\forall t=1,\ldots, T\label{const:upos2}	\\
    % &&v^t_{jr}			&\geq 0 															&\forall j &\in \dc, \forall r\in \rod,\forall t=1,\ldots, T\label{const:vpos}	\\
    &&I_{it}			&\geq 0 															&\forall i &\in \loc, \forall t=1,\ldots, T\label{const:invCpos2}	\\
    %&&I^t_{j}			&\geq 0 															&\forall j &\in \dc, \forall t=1,\ldots, T\label{const:invDpos}	\\
    &&y_{j}					& \in \{0,1\} 														&\forall j&\in \loc\setminus\cu\label{const:ybool2}\\	
    &&z_{rt}					&\in \{0,1\} 														&\forall r&\in \ro, \forall t=1,\ldots,T\label{const:zbool2}
\end{alignat}

\subsection{Restrictions}
Although the MIP formulation from~\S~\ref{subsection:MIP} deals with a rather general version of the problem, we focus on two particular cases:
\begin{enumerate}
    \item The first one considers a model ``Direct+Loop'' in which the set $\ro_k$ consists only of direct routes between the supplier and the depots. 
        That is, a route $r\in\ro_k$ can be described by two arcs $(s,j)$ and $(j,s)$, where $j\in\ro_k$ and $s$ is the supplier.
    \item The second one, called ``Loop+Direct'' considers a set $\ro_L$ containing only direct routes between depots and customers. 
        In other words, a route $r\in\ro_L$ can be described by two arcs $(j,i)$ and $(i,j)$ with $j\in\loc_k$ and $i\in\cu$. Figure~\ref{fig:ld}.
\end{enumerate}
Figure~\ref{fig:dl} and~\ref{fig:ld} illustrates the two types of structures related to these models.
\begin{figure}
    \centering
    \subcaptionbox[.45\linewidth]{Model ``Direct-Loop''\label{fig:dl}}{\inputfig{model1}}
    \subcaptionbox[.45\linewidth]{Model ``Loop-Direct''\label{fig:ld}}{\inputfig{model2}}
    \caption{The two types of routing structures considered}
\end{figure}


\section{Matheuristic method} 

The matheuristic method we propose for the LIRP is based on a sampling of the routes. 
After being shuffled, the original set of routes is splitted into $\alpha$ independent subsets. 
Each subset is solved with Cplex, leading to $\alpha$ suboptimal LIRP solutions.  
The routes contained in each suboptimal solution are stored in a pool of \textit{promising routes}. 
Finally, the pool of promising routes is used to create a new (hopefully tractable) instance which solved with Cplex. 
Algorithm \ref{mainalgo} describes the procedure.


\begin{algorithm}
	\caption{The proposed sampling matheuristic}
	\label{mainalgo}
	%\scriptsize
	\begin{algorithmic}[1]
		\REQUIRE  The set $R$ of all routes generated
		\REQUIRE A parameter $\alpha$ (number of subsets)
		\REQUIRE $t1$ and $t2$: CPU allocated to Cplex in the inner and outer loop of the algorithm respectively
		\REQUIRE $J$ : parameter
		\STATE $z^*= +\infty$
		\STATE $collectedRoutes \leftarrow \emptyset$
		\STATE $preA = preAllocate()$					 \COMMENT{Pre-allocation of clients to depots}
		\REPEAT
			\STATE $amelioration = FALSE$
			\STATE $R =  R \backslash collectedRoutes$
			\STATE Shuffle all routes in $R$
			\STATE $S \leftarrow SelectLocations(preA, \alpha)$ \COMMENT{selects $\alpha$ routes}
			\STATE $R' = RouteFiltering(R,S)$
			\STATE $RouteSampling(R',\alpha)$	\COMMENT{partition $R'$ into $\alpha$ independent subsets of routes}
			\FOR{$s=1$ to $\alpha$}
				\STATE $R_s = $ routes in subset $s$
				\STATE $bestRoutes_s = \emptyset$
				\FOR {$j=1$ to $J$}
					\STATE $R_s \leftarrow R_s \backslash bestRoutes_s$ \COMMENT{remove the route just found, active when $J>1$}
					\STATE $(z,bestRoutes_s) = SolveLIRP(R_s,t1)$ 
					\STATE $collectedRoutes \leftarrow bestRoutes_s$
				\ENDFOR
			\ENDFOR
			\STATE 	$(z,bestRoutes) = solveLIRP(collectedRoute, t2)$
			\IF {$z < z^*$}
				\STATE $z^*=z$
				\STATE $amelioration =TRUE$
			\ENDIF
		\UNTIL {$amelioration = FALSE$}
		\RETURN $bestRoutes$
	\end{algorithmic}
\end{algorithm}


\subsection{Location based sampling}

Location Sampling s'effectue en deux temps. 
Tout d'abord, les clients sont pré-affectés à un certain nombre de dépôts, comme décrit dans l'algo \ref{algo:pre-allocation}.
Ensuite, cette pré-affectation est utilisée pour constituer des sous-ensembles de routes, comme décrit dans l'algo 	\ref{algo:location-sampling}



\begin{algorithm}
	\caption{Pre-allocations of clients to depots}
	\label{algo:pre-allocation}
	%\scriptsize
	\begin{algorithmic}[1]
		\REQUIRE  $D$: set of depots
		\REQUIRE $C$: set of clients
		\REQUIRE $N_{close}, \mu_1, \mu_2, \mu_3$: parameters 
		\STATE
		\STATE Initialization: $preA$: $|C| \times |D|$ pre-allocation matrix filled with zeros
		\STATE Initialization: $dist$: $|C|\times (|D|+|C|)$ distance matrix between clients and other clients + depot
		\STATE
		\FOR {all clients $c \in C$}
			\FOR {all depots $d \in D$}
				\STATE calculate the distance $dist(c,d)$ between $c$ and $d$
			\ENDFOR
		\ENDFOR
		\STATE
		\FOR {all clients $c \in C$}
			\STATE $ClosestDepots(c) = $ List of all depots $d \in D$ ranked in non-decreasing order of the distances $dist(c,d)$.
			\STATE 	$A(c,ClosestDepots(c)[1]) =1$ 	\COMMENT{pre-allocate client $c$ to its closest depot}
			\FOR {$n=2$ to $N_{close}$}
				\IF {$dist(c,ClosestDepots(c)[n]) < \mu_1$}
					\STATE 	$A(c,ClosestDepots(c)[n]) =1$ 	\COMMENT{pre-allocate client $c$ to depot ranked $n$}
				\ENDIF
			\ENDFOR
		\ENDFOR
		\STATE
		\FOR {all clients $c \in C$ }
			\STATE Rank all clients $c' \in C$, $c' \neq c$ in non-decreasing order of the distances $dist(c,c')$ 
			\FOR {all clients $c'$}
				\IF {$dist(c,c') < \mu_2$} 
					\FOR {all $d' \in ClosestDepots(c')$}
						\IF {$A(c',d')=1$ and $preA(c,d')=0$ and $dist(d',c')+dist(c',c)+dist(c,d') < \mu_3$}
							\STATE $preA(c,d') =1$
						\ENDIF
					\ENDFOR
				\ENDIF
			\ENDFOR
		\ENDFOR
		\RETURN $preA$
	\end{algorithmic}
\end{algorithm}


Lines 5--9 of Algorithm \ref{algo:pre-allocation}, calculate distances between each client and all depots. For each client, these distances are ranked in non-decreasing order (line 12). Then, all clients are pre-allocated to their closest depot (line 13) and to 
the $N_{close}$ closest depots provided the distance between the client and the depot does not exceed value $\mu_1$. 

In lines 21--32 of Algorithm \ref{algo:pre-allocation} deal with the special case where two neighbouring clients are not allocated to the same depots. 
Lines If we detect two clients $c$ and $c'$ such that 
$i)$ distance between $c$ and $c'$ is no more than value $\mu_2$ (line 24), 
$ii)$ $c$ is allocated to some depot $d$ (line 26)
$iii)$ $c'$ is pre-allocated to some depot $d' \ne d$ (line 26),
$iv)$ $dist(d,c')+dist(c',c)+dist(c,d') < \mu_3$ (line 26),  
then $c$ is pre-allocated to $d'$ (line 27).  



%------------------------------------------------------------

\begin{algorithm}
	\caption{Heuristic selection of depots}
	\label{algo:location-sampling}
	%\scriptsize
	\begin{algorithmic}[1]
		\REQUIRE $D$: set of depots
		\REQUIRE $C$: set of clients
		\REQUIRE $preA$: pre-allocation matrix
		\REQUIRE $p$: number of depots to be selected
		\REQUIRE $\beta$: parameter strictly greater than 1.
		\STATE Initialize: fill the $|C| \times |D|$ allocation matrix $A$ with zeros
		\STATE $nbClient \leftarrow 0$ number of clients allocated to some depot
		\STATE $S \leftarrow \emptyset$: list of selected depots
		\FOR {each depot $d \in D$}
			\STATE $score(d) =$ number of clients pre-allocated to $d$
		\ENDFOR
		\WHILE {$|S| < p$ and $nbClient < |C|$ } 
			\STATE Rank all depots by non-increasing scores
			\STATE Choose a random number $y$ from the interval $[0,1)$
			\STATE $d^{new} = D[\;\lceil y^\beta |D| \rceil\;]$
			\STATE $ S = S \cup \{d^{new}\}$.
			\FOR {all clients $c \in C$}
				\IF {$preA[c][d^{new}] =1$}
					\STATE $A[c][d^{new}] = 1$
					\STATE $nbClient \leftarrow nbClient +1$
					\FOR {all depots $d\ne d^{new}$}
						\IF {$preA[c][d] = 1$}
						\STATE $preA[c][d] \leftarrow 0$
						\STATE $score(d) \leftarrow score(d)-1$
						\ENDIF
					\ENDFOR
				\ENDIF		
			\ENDFOR
		\ENDWHILE
		\IF {$nbClient < |C|$}
			\STATE Add a dummy depot $\delta$
			\FOR {all $c\in C$ such that $\sum\limits_{d \in D} preA[c][d] >0$}
				\STATE $A[c][\delta]=1$
			\ENDFOR
		\ENDIF
		\RETURN $S$
		\RETURN $A$
	\end{algorithmic}
\end{algorithm}

Algorithm \label{algo:location-sampling} takes the pre-allocation matrix $preA$
as an input, selects $p$ depots and builds the allocation matrix $A$.
The selection of depots is based on a score. 
For each depot, this score corresponds to the number of clients pre-allocated to it (lines 4--6). 

The algorithm is an iterative process based on a roulette wheel selection of the next depot. On lines 8--10, the depots are ranked in non-increasing order of their scores. Then depot position at rank  $\lceil y^\beta |D| \rceil$ is selected, where $y$ is a random value in the interval $[0,1)$ and $\beta$ a parameter strictly greater than 1. 

Once a depot, called $d^{new}$, has been selected, all clients pre-allocated to $d^{new}$ are allocated to it. Thus, their pre-allocation to all other depots is erased (line 18) and the score of other depots is updated (line 19). 

The main loop (lines 7--24) stops as soon as $p$ depots have been selected or all clients have been allocated to some depot. If $p$ depots have been selected and some clients are still not allocated, the remaining clients are allocated to a dummy depot $\delta$ (lines 25--30). The dummy depot has an arbitrarily large fixed cost and we assume there is a direct route from $\delta$ to all clients. This dummy depot is used to guarantee the existence of a ``feasible'' solution.   



\newpage

After selecting a subset of depots, we filter the list of existing routes by removing all routes visiting unselected depots. 

\subsection{Filtering SD routes after depot selection}

Algorithm  \ref{algo:SDfiltering} filters routes finishing at unselected depots. 


\begin{algorithm}
	\caption{SD Routes filtering}
	\label{algo:SDfiltering}
	%\scriptsize
	\begin{algorithmic}[1]
		\REQUIRE $SDloop$: set of SD routes
		\REQUIRE $D$ set of depots
		\REQUIRE $C$: set of clients
		\REQUIRE $S$: set of selected depots
		
		\STATE $filteredSD \leftarrow \emptyset$
		
		\FOR {all $d\in D$}
				\IF {$d \in S$}
					\STATE $filteredSD = filteredSD \cup \{s,d\}$
				\ENDIF
		\ENDFOR
		
		\RETURN $filteredSD$
	\end{algorithmic}
\end{algorithm}



\subsection{Filtering DC routes after depot selection}

Algorithm  and \ref{algo:DCfiltering} filters routes starting from unselected depots. 
Routes that start from an unselected depot (line 4), 
or that contain clients that are not pre-allocated to the starting depot of the route (line 6) are filtered. 

\begin{algorithm}
	\caption{DC Routes filtering}
	\label{algo:DCfiltering}
	%\scriptsize
	\begin{algorithmic}[1]
		\REQUIRE $DCloop$: set of DC routes
		\REQUIRE $preA$: pre-allocation matrix
		\REQUIRE $C$: set of clients
		\REQUIRE $S$ set of selected depots
		
		\STATE $filteredDC \leftarrow \emptyset$
				
		\FOR {each route $r \in DCloop$}
			\STATE $keep = 1$
			\IF {$depot(r) \in S$}
				\FOR {all $c \in r$}	
					\IF{$preA[c][depot(r)]=0$}
						\STATE $keep=0$
						\STATE $c \leftarrow $ end of the route
					\ENDIF
				\ENDFOR
			\ELSE 
				\STATE{$keep=0$}  
			\ENDIF
			\IF {$keep=1$} 
				\STATE $filteredDC =filteredDC \cup \{r \}$
			\ENDIF
		\ENDFOR
		\RETURN $filtered DC$
	\end{algorithmic}
\end{algorithm}



%\begin{algorithm}
%	\caption{Generation of location oriented subsets (old version)}
%	\label{algo:location-sampling2}
%	%\scriptsize
%	\begin{algorithmic}[1]
%		\REQUIRE  $D$: set of depots
%		\REQUIRE $\alpha$: number of subsets to be generated
%		\REQUIRE $A$: pre-allocation matrix
%		\REQUIRE $\alpha$: number of subsets
%		\IF {$\alpha \ge |D|$ } 
%		\STATE $ListOfSubsets = \emptyset$
%		\STATE
%		\FOR {depot index $d=1$ to $|D|$} 
%			\STATE $ListOfSubsets = ListOfSubsets \cup \{d\}$
%		\ENDFOR
%		\STATE $NbSubsets \leftarrow |D| $		
%		\STATE
%		\STATE $n \leftarrow 2$
%		\WHILE{$NbSubsets < \alpha$}
%			\STATE $S_n$ = set of all sets of $n$ depots, ranked in non-increasing order of the number of common pre-allocated clients.
%			\REPEAT 
%				\STATE $s =$ first element of $S_n$
%				\STATE $ListOfSubsets = ListOfSubsets \cup \{s\}$
%				\STATE Remove $s$ from $S_n$
%				\STATE $NbSubsets \leftarrow NbSubsets+1$
%			\UNTIL ($S_n = \emptyset$ or $NbSubsets = \alpha$)
%		\STATE 	$n \leftarrow n+1$
%		\ENDWHILE
%		\ENDIF
%		\STATE 
%		\IF {$\alpha <  |D|$}
%			\FOR {all pairs $(i,j)$ of depots} 
%			\STATE Compute the similarity measure $M(i,j)$ equal to the number of clients pre-allocated both to $i$ and $j$.
%			\ENDFOR
%			\STATE Run $|D| - \alpha - 1$ iterations of Kruskal's algorithm with weights $\max\limits_{i,j} M(i,j) - M(i,j)$
%			\STATE $ListOfSubsets$ = one subset for each subtree yielded by Kruskal's algorithm 
%		\ENDIF
%		\RETURN $ListOfSubsets$
%	\end{algorithmic}
%\end{algorithm}


\newpage

\bibliographystyle{plain}
\bibliography{biblio_LIRP}

\end{document}
